\chapter{Outlook}\label{chap:outlook}

Orbit Tomography is a powerful new technique; its proliferation as a common diagnostic tool, however, is uncertain. What is certain is that without continual improvements to the method and new applications, this promising technique will die on the vine. In the following few sections, we discuss how the technique can be improved and a new application to runaway electrons.

\section{Combining Multiple Fast-ion Diagnostics}
While the demonstrated application of Orbit Tomography limited itself to FIDA data, there is no reason why data from other fast-ion diagnostics could not also be incorporated into the reconstructions. In fact, most devices have more than one set of fast-ion diagnostics. For example, MAST-U has several fast-ion diagnostics: FIDA spectrometers, neutron collimators, 3 MeV proton detectors, and NPAs. In order to incorporate a new diagnostic, the orbit weight functions need to be calculated, which requires a complete forward model of the diagnostic.

Orbit weight functions have already been calculated for the beam-plasma neutron rate and for DIII-D's solid state NPA. The reason they were not included in the reconstruction of DIII-D shot \#171469 is that they lack an absolute calibration, which is necessary for Orbit Tomography. In the future this limitation could be bypassed by ``calibrating'' the diagnostic to a reference discharge. This is the planned approach for incorporating the Imaging NPA (INPA) diagnostic into the reconstructions.

In principle, adding new diagnostic data is as simple as adding a row to the weight matrix, but in practice there may be complications. One complication that was observed when combining diagnostics in Velocity-space Tomography\cite{salewski2012,salewski2013_tomography,salewski2017mev,salewski2018bayesian} was that data from a single diagnostic can dominate the reconstruction. This is undesirable since the reconstruction can, to its detriment, ignore the data from other diagnostics. This can lead to over-fitting. It is expected that this will occur when the INPA is incorporated since it provides many, low-noise measurements. If any biases or systematic errors are present in the INPA data, then the other diagnostics will not be able to compensate.

\section{Locally Smooth Covariance Matrix}
Presently, Orbit Tomography uses a Gaussian Process prior (Eq. \ref{eq:prior}) with a squared exponential kernel (Eq. \ref{eq:kernel}) to correlate different points in orbit-space. This has a smoothing effect. However, the benchmark performed in the previous chapter showed that this choice of prior produced distributions that were \emph{too} smooth, causing peaked features to be washed out. The reason for this is the flawed assumption that orbits that are near each other in orbit-space should be more correlated than points that are far away from each other, i.e. we assume that the distribution is smooth in orbit-space. This is flawed because the physical process that causes distribution functions to be smooth, scattering, occurs between particles, not orbits. Apart from sharp gradients that occur near loss boundaries, the \emph{know} distribution is smooth in \{E,p,r,z\}-space, since that is the space where scattering occurs. 

To properly correlate the orbits, we define a covariance matrix in \{E,p,r,z\}-space, $\mathbf{\Sigma}_{\{E,p,r,z\}}$, and then transform it into a covariance matrix in orbit-space, $\mathbf{\Sigma_n}$.
The linear transformation of a covariance matrix is given by
\begin{equation}\label{eq:transform_covariance}
    \mathbf{\Sigma_n} = \mathbf{R} \cdot \mathbf{\Sigma}_{\{E,p,r,z\}} \cdot \mathbf{R}^T\,,
\end{equation}
where $\mathbf{R}$ is a matrix that converts a discretized distribution in \{E,p,r,z\}-space, $\mathbf{f}$, into an orbit-space distribution, $\mathbf{n} = \mathbf{R} \cdot \mathbf{f}$.

Unlike in orbit-space, a squared exponential kernel is appropriate in \{E,p,r,z\}-space. However, computer memory becomes an issue if the \{E,p,r,z\}-space grid is large, as $\mathbf{\Sigma}_{\{E,p,r,z\}}$ is a $(n_E \cdot n_p \cdot n_r \cdot n_z) \times (n_E \cdot n_p \cdot n_r \cdot n_z)$ matrix, where $n_X$ denotes the number of elements in the $X$ dimension. It also becomes computationally inefficient since matrix multiplication scales as $\mathcal{O}(N^3)$ where $N$ is the size of the matrix. 
Fortunately, we can use the fact that orbits naturally correlate different spatial locations together to create a covariance matrix that is both memory and computationally efficient.

Let us define an $(n_E \cdot n_p) \times (n_E \cdot n_p)$ \{E,p\}-space covariance matrix for every point in \{r,z\}-space, the elements of which are given by
\begin{equation}\label{eq:ep_covariance}
    {\Sigma_{\{E,p\}}}_{ij} = \theta_1^2 \exp \left (-\frac{1}{2}\left ( ([E,p]_i - [E, p]_j)^T\cdot \rm{diag}(\theta_2^2,\theta_3^2)^{-1} \cdot ([E, p]_i - [E, p]_j)\right ) \right )\,,
\end{equation}
where $\theta_{1-3}$ are hyper-parameters controlling the amount of correlation/smoothness. 
Since we are letting the transformation into orbit-space correlate different points in \{r,z\}-space, we can define the \{E,p,r,z\}-space covariance matrix to be, 
\begin{equation}\label{eq:eprz_covariance}
    {\Sigma_{\{E,p,r,z\}}}_{ij} = ({\delta_r}_{ij} \cdot {\delta_z}_{ij}) \mathbf{\Sigma}_{\{E,p\}}\,,
\end{equation}
where ${\delta_r}_{ij}$ and ${\delta_z}_{ij}$ are Kronecker deltas whose product only take the value of one when the $i^{th}$ and $j^{th}$ points have the same r \emph{and} z values. This forms a block diagonal matrix where each block is identical. We can represent this compactly as a Kronecker product,
\begin{equation}\label{eq:eprz_covariance_kron}
    \mathbf{\Sigma}_{\{E,p,r,z\}} = \mathbf{I} \otimes \mathbf{\Sigma}_{\{E,p\}}\,,
\end{equation}
where $\mathbf{I}$ is a $(n_r \cdot n_z) \times (n_r \cdot n_z)$ identity matrix and $\otimes$ denotes a Kronecker product. Since this matrix is sparse, it can be multiplied efficiently and  stored compactly.
Another advantage of this choice of covariance matrix is that the inverse of the matrix can also be represented as a Kronecker product,
\begin{equation}\label{eq:block_diag_inv}
    \mathbf{\Sigma}_{\{E,p,r,z\}}^{-1} = \mathbf{I} \otimes \mathbf{\Sigma}_{\{E,p\}}^{-1} \,.
\end{equation}
This property makes the prospect of inferring the distribution function in \{E,p,r,z\}-space more feasible.

\section{Application to Runaway-electron Diagnostics}
Although this thesis focuses primarily on fast ions, there is a second energetic particle species that is becoming an increasingly important area of research, the runaway electrons.
During a disruption event an electric field is generated, causing thermal electrons to reach relativistic energies.
If the energy of the electrons exceed a critical energy, the acceleration due to the electric field exceeds the deceleration due to collisional drag and a runaway effect occurs.
If containment is lost, these runaway electrons can severely damage the vessel walls and cooling systems.
Wall damage can adversely impact subsequent experiments and should be repaired as soon as possible.
For most devices, however, vessel access is limited and repairs are often delayed.
This is especially true for ITER where repairs are expected to take months due to the complex machinery and the expected strong irradiation of plasma facing components.\cite{iter1999overview,ikeda2007progress,hender2007mhd,boozer2017runaway}
It is no wonder then that development of runaway electron control mechanisms are becoming a focus of the fusion community.
 
A key challenge of developing effective control mechanisms is simply being able to measure the runaway-electron distribution function in its entirety---imagine trying to control something you cannot fully see.
Since runaway electrons, like fast ions, follow orbits, the principles and techniques of fast-ion diagnostics can be directly applied to runaway-electron diagnostics. Provided there is a complete forward model of the runaway electron diagnostics, we can use Orbit Tomography to infer the runaway electron distribution function. 

There are many different runaway-electron diagnostics: scintillators, synchrotron cameras and spectrometers; however, the recently developed Gamma Ray Imager\cite{paz2017spatiotemporal} (GRI) shows the most promise for Orbit Tomography. The Gamma Ray Imager measures bremsstrahlung, hard-x-ray emission produced when the runaway electrons are scattered off the background plasma.
The diagnostic provides spatial, temporal, energy, and---to a lesser extent---pitch angle resolved measurements; a diagnostic built for Orbit Tomography. Plans have been made to calculate the Gamma Ray Imager's orbit weight functions and work will begin in the coming months.
